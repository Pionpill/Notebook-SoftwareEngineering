\subsection{Prototype (原型)}

\noindent\textbf{意图}

用原型实例指定创建对象的种类,并且通过拷贝这些原型创建新的对象。

\noindent\textbf{动机}

如果我们需要创建同一个类的多个实例,这需要给这些实例分配大量的内存,消耗 CPU 资源。而在有些情况下,往往这些实例的属性是相同或者只有极小的差别。

为了释放内存资源,提高性能,我们采用原型模式,直接在内存中拷贝已有的实例。

\noindent\textbf{适用性}

下列情况可以使用 Prototype 模式:
\begin{itemize}
    \item 一个系统独立于它的产品创建,构成和表示时。
    \item 要实例化的类是在运行时指定的,例如动态转载。
    \item 避免创建一个与产品类层次平行的工厂层次时。
    \item 一个类的实例只有几个不同状态中的一种时。建立原型并克隆他们。
\end{itemize}

\noindent\textbf{结构}

\begin{figure}[H]
    \scriptsize
    \centering
    \begin{tikzpicture}[scale = 1]
        \begin{class}[text width=2.5cm]{Client}{0,0}
            \operation{Operation()}
        \end{class}
        \begin{interface}[text width=2.5cm]{Prototype}{7,0}
            \operation[0]{Clone()}
        \end{interface}
        \draw[umlcd style,->] (Client) -- (Prototype);
        \begin{class}[text width=3cm]{ConcretePrototypr1}{4,-3}
            \implement{Prototype}
            \operation{Clone():return copy of self}
        \end{class}
        \begin{class}[text width=3cm]{ConcretePrototypr2}{10,-3}
            \implement{Prototype}
            \operation{Clone():return copy of self}
        \end{class}
    \end{tikzpicture}
\end{figure}

\noindent\textbf{参与者}
\begin{itemize}
    \item \textbf{Prototype}: 声明一个克隆自身的接口
    \item \textbf{ConcretePrototype}: 实现一个克隆自身的操作
    \item \textbf{Client}: 让原型克隆自身从而创建一个新的对象
\end{itemize}

\noindent\textbf{协作}

\begin{itemize}
    \item 客户请求一个原型克隆自身。
\end{itemize}

\noindent\textbf{优缺点}

\begin{itemize}
    \item \textbf{运行时增加和删除产品}: 允许客户注册原型实例就将一个新的具体产品类并入系统。比其它创建型模式更灵活。
    \item \textbf{减少子类构造}
    \item \textbf{动态配置应用}
\end{itemize}

\noindent\textbf{实现}

\begin{itemize}
    \item \textbf{使用原型管理器}: 当一个系统中原型数目不固定时,要保持一个可用原型的注册表。在注册表中存储和检索原型,我们通常称之为原型管理器(Prototype Manager)。
    \item \textbf{实现克隆操作}: Prototype 模式最难实现的地方在于实现 Clone 操作。当对象包含循环时,者尤其棘手。
\end{itemize}

\noindent\textbf{相关模式}

\begin{itemize}
    \item \textbf{Abstract Factory}: 在某些方面,这两个模式是相互竞争的,但他们也可以一起使用。Abstract Factory 可以存储一个被克隆的原型的集合,并返回产品对象。
    \item \textbf{Composite/Decorator}: 这两个设计模式可以从 Prototype 中获益。
\end{itemize}

\noindent\textbf{代码}

\begin{itemize}
    \item Java 实现 Prototype: \url{https://blog.csdn.net/qq_38526573/article/details/87633257}
\end{itemize}

\lstinputlisting[language=Python]{../../../scripts/creational/Prototype.py}

\newpage