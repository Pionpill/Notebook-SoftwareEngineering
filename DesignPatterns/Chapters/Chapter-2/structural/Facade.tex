\subsection{Facade (外观)}

\noindent\textbf{意图}

为子系统中的一组接口提供一个一致的界面,Facade 模式定义了一个高层接口,这个接口使得这一子系统更加容易使用。

\noindent\textbf{动机}

将一个系统划分成若干个子系统有利于降低系统的复杂性。一个常见的设计目标时使子系统间的通信和相互依赖关系达到最小。达到该目标的途径之一就是引入一个外观对象,它为子系统中较一般的设施提供了一个单一而简单的界面。

\noindent\textbf{适用性}

\begin{itemize}
    \item 要为一个复杂子系统提供一个简单接口时。子系统往往因为不断演化而变得越来越复杂,大多数模式使用时都会产生更多更小的类。这使得子系统更具可复用性,也更容易对系统进行定制。
    \item 客户程序与抽象类的实现部分之间存在着很大的依赖性。引入 Facade 将这个子系统与客户以及其他的子系统分离,可以提高子系统的独立性与可移植性。
    \item 当你需要构建一个层次结构的子系统时,使用 Facade 模式定义子系统中每层的入口点。如果子系统之间是相互依赖的,可以让它们仅通过 Facade 进行通信,从而简化了它们之间的依赖关系。
\end{itemize}

\noindent\textbf{结构}

\begin{figure}[H]
    \scriptsize
    \centering
    \begin{tikzpicture}[scale = 1,umlcd style]
        \draw[dashed] (-5,-4) rectangle (5,0);
        \begin{class}[text width=2cm]{Facade}{0,0.5}
        \end{class}
        \begin{class}[text width=2cm]{ModuleA}{-3,-1.2}
        \end{class}
        \begin{class}[text width=2cm]{ModuleB}{-1,-2.5}
        \end{class}
        \begin{class}[text width=2cm]{ModuleC}{2,-2}
        \end{class}
        \draw[umlcd style,->] (ModuleA) -- (Facade);
        \draw[umlcd style,->] (ModuleB) -- (Facade);
        \draw[umlcd style,->] (ModuleC) -- (Facade);
    \end{tikzpicture}
\end{figure}

\noindent\textbf{参与者}

\begin{itemize}
    \item \textbf{Facade}: 知道哪些子系统类负责处理请求;将客户的请求代理给适当的子系统对象。
    \item \textbf{Subsystem classed}: 实现子系统的功能;处理 Facade 对象指派的任务;没有 Facade 的任何相关信息,即没有指向 Facade 的指针。
\end{itemize}

\noindent\textbf{协作}

\begin{itemize}
    \item 客户程序通过发送请求黑 Facade 的方式与子系统通信,Facade 将这些消息转发给适当的子系统对象。尽管是子系统中的有关对象在做实际工作,但 Facade 模式本身也必须将它的接口转换成子系统的接口。
    \item 使用 Facade 的客户端程序不需要直接访问子系统对象。
\end{itemize}

\noindent\textbf{优缺点}

\begin{itemize}
    \item 对客户屏蔽了子系统组件,减少了客户处理对象的数目并使得系统用起来更加方便。
    \item 是西安了子系统与客户之间的松耦合关系,而子系统内部的功能组件往往是紧耦合的。
\end{itemize}

\noindent\textbf{实现}

\begin{itemize}
    \item \textbf{降低客户-子系统之间的耦合度}: 用抽象类实现 Facade 而它的具体子类对应于不同的子系统是心啊,这可以进一步降低客户与子系统的耦合度。
    \item \textbf{公共子系统与私有子系统}: 一个子系统与一个类的相似之处是,它们都有接口并且都封装了一些东西。而子系统封装了一些类。子系统的公共接口包含所有客户程序都可以访问的类,私有接口仅用于对子系统进行扩充。
\end{itemize}

\noindent\textbf{例子}

\begin{itemize}
    \item Java: \url{https://blog.csdn.net/qq_45034708/article/details/114972361}
    \item Video: \url{https://www.bilibili.com/video/BV1xz4y1X7HW}
\end{itemize}

\lstinputlisting[language=Python]{../../../scripts/structural/Facade.py}

\newpage